\chapter{Einführung in Mainframe}

%%%%%%%%%%%%%%%%%%%%%%%%%%%%%%%%%%%%%%%%%%%%%%%%%%
\section{Z Systems}
Im Rahmen dieser Bachelorarbeit wird keine dezentrale Software entwickelt, sondern zentral auf dem IBM Mainframe (früher Großrechner) gearbeitet.  In der \FirmenName wird das Flaggschiff von IBM eingesetzt, das \emph{Z~Systems}, aktuell in der 15. Generation. Der Mainframe ist eine besondere Art Computer, die mitunter ähnliche Aufgaben erfüllt wie ein Server. Jedoch treten zum Teil wesentliche Unterschiede auf.

\subsection{Was ist ein Mainframe}

Der Mainframe ist eine Art Computer, jedoch wesentlich größer, daher auch der alte Name \emph{Großrechner}. Durch seine besonderen Eigenschaften ist er für geschäftskritische Tätigkeiten geeignet. Die \ac{hw} ist intern redundant, was eine hohe Ausfallsicherheit bietet. Die \ac{hw} kann nach Bedarf nahtlos erweitert werden und ausgeprägtes Workloadbalancing ermöglicht eine effiziente Verarbeitung großer Datenmengen. Mainframe-\ac{sw} bietet zusätzliche Funktionen zur Sammlung und anschließenden Abarbeitung von Aufgaben. Diese Eigenschaften befähigen den Mainfame zu \ac{olap}, \ac{oltp} und Batchverarbeitung.
% Für die \ac{edv} im Businesskontext können verschiedene Technologien eingesetzt werden. Der \emph{Mainframe} >>ist ein zentrales Datenlager<<\cite{redbook.1}, welcher zur konkurrierende Datenverarbeitung eingesetzt wird. Er ist besonders geeignet für Prozesse, die einen hohen Datendurchsatz haben, z.B. \ac{oltp}, \ac{olap} und Batch.


\subsubsection[Z Systems vs. x64]{Unterschied zwischen Mainframe und x64}

Möchte man im großen Stil Daten verarbeiten oder Software abspielen, so ist ein Server die einfachste Lösung. Ein klassischer Server ist ein Computer mit derselben Struktur wie ein herkömmlicher PC nach der \mbox{von-Neumann-Architektur}. Wie ein PC ist er vielseitig und einfach zu betreiben.

Ein \emph{Mainframe}, zu Deutsch Großrechner, verfügt dagegen über andere Komponenten und eine andere Architektur. Er ist im Vergleich viel spezialisierter und wird nur für sehr konkrete Aufgaben verwendet, da es schwieriger ist, ein Mainframe zu programmieren und zu betreiben.
\begin{table}[H]
    \centering
    \begin{tabularx}{\textwidth}{|X|X|}
        \hline
        \textbf{Server} & \textbf{Mainframe} \\
        \hline
        Server verfügen nur über einfache Komponenten. Diese sind auf dem Markt in großer Stückzahl vorhanden und dadurch günstig in der Anschaffung.
        &
        Weil sie aus besonderen Komponenten bestehen, die nur bei wenigen Herstellern verfügbar sind, sind Mainframes teuer in der Anschaffung. Heute sind >\,99 \% aller Mainframes von IBM.
        \\
        \hline
        Die \ac{hw} ist allgemein und vielseitig einsetzbar. Interne Prozesse sind in \ac{sw} implementiert. Selbst dedizierte \ac{hw}, wie eine Graphical Processing Unit oder eine Tensor Processing Unit, ist nicht auf einen bestimmten Anwendungsfall beschränkt.
        &
        Die \ac{hw} ist komplex und spezialisiert. Für jede Art von Operation, z.B. \ac{os}, In-/Output, Datenbank-\ac{sw} oder Java-Applikation \cite{redbook.1}, ist eigene \ac{hw} vorhanden.
        \\
        \hline
        Server können flexibel für jede Aufgabe eingesetzt. Server sind besser bei benutzerspezifischen Anwendungen mit hoher Interaktion.
        &
        Spezialisierte \ac{hw} führt zu einem hohen Datendurchsatz. \ac{oltp}, z.B. Banktransaktionen, und \ac{olap}, z.B. statistische Analysen auf dem \ac{dwh}, laufen wesentlich schneller.
        \\
        \hline
        Im Vergleich zu PCs verfügen Server über eine starke Netzwerkkarte und können so miteinander verbunden werden. Ein kleiner Verbund wird als \emph{Servercluster}, ein großer Verbund als \emph{Serverfarm} bezeichnet. Für einen leistungsstarken Verbund wird eine hohe Anzahl an Servern benötigt.
        &
        Ähnlich der Funktion einer Netzwerkkarte, verfügt ein Mainframe über ein \ac{cf}. Das ist eine Kombination aus verschiedenen Prozessortypen, die Daten cacht, logische und physische Adressen übersetzt und .
        \\
        \hline
        Downtime Ø >\,0,0001~\%
        &
        Downtime <\,0,0000001~\%~\cite{itic}, daher auch >Z< für >>Zero Downtime<<
        \\
        \hline
    \end{tabularx}
    \caption{Meine Tabelle}
\end{table}


\subsection{Aufbau Mainframe}

Obwohl ein Mainframe größer und leistungsstärker ist als ein Server, ist noch immer nicht leistungsstark genug, um die gesamte \ac{edv} eines Unternehmens zu beherrschen. Deswegen werden Mainframes ähnlich wie Server skaliert. Die beiden größten Skalierungsmethoden sind der physische Zusammenschluss mehrerer Maschinen und die logische Aufteilung zur Lastverteilung.


\subsubsection{Sysplex}
\todo{Parallel Sysplex nur mit \z?}
Ein Parallel \ac{sys}, oder einfach nur \ac{sys}, besteht aus einem oder mehreren Mainframe-Maschinen, die >>durch spezialisierte \ac{hw} zu einer Einheit zusammengeschlossen sind<<. Innerhalb eines \ac{sys} können Mainframes Ressourcen teilen, z.B. Hauptspeicher, und I/O - Schnittstellen. Über \acp{cf} ist eine Maschine mit anderen verbunden. Ein \ac{sys} ist als ein einzelnes System zu betrachten. Prozesse können von einer beliebigen Kombination von Maschinen innerhalb des \ac{sys} ausgeführt werden, wodurch die Arbeitslast immer gleichmäßig verteilt wird und bis in die hohe Auslastung skaliert. Während Server bereits bei 20 \% Auslastung langsamer werden, können Z Systems bis >\,90~\% belastet werden. Sind alle Anwendungen für den Parallel \ac{sys} optimiert, >>kann die Arbeitslast:
\begin{itemize}
    \item dynamisch über alle Systeme verteilt werden
    \item vertikal und horizontal skaliert werden
    \item bei geplantem und ungeplantem Ausfall eines Systems verfügbar bleiben
    \item alle Systeme innerhalb des Setups als eins benutzen<<\cite{redbook.1}
\end{itemize}

In der \FirmenName werden mehrere \acp{sys} verwendet. So gibt es jeweils unterschiedliche \acp{sys} für Produktions- und Entwicklungsumgebung. Während dieser Bachelorarbeit wird ausschließlich in einer Testumgebung geareitet. Der \textbf{E-Plex} wird für Entwicklung, Test und Schulungen verwendet. Er enthält keine Produktionsdaten oder ist nicht kritisch für den Betrieb. Dies ist wichtig, um den laufenden Betrieb des Unternehmens nicht zu stören oder gar zu stoppen. 


\subsubsection{LPAR}

Ähnlich wie virtuelle Maschinen auf einem Servercluster ausgeführt werden, wird die \ac{hw} des Mainframes in logische Gruppen unterteilt. Eine >>\ac{lpar} ist eine Gruppe an \ac{hw}-Prozessoren, die ein \ac{os} ausführt<<\cite{redbook.1}. Die Unterteilung des physischen Systems wird über einen Hypervisor Typ~1 realisiert, eine Firmware, die in den Prozessor eingebaut ist und nicht Teil des \ac{os} ist. Jede \ac{lpar} führt ihr eigenes \ac{os} und Scheduling aus.

%%%%%%%%%%%%%%%%%%%%%%%%%%%%%%%%%%%%%%%%%%%%%%%%%%
\section[\z]{Betriebssystem \z}
Das \ac{os} \emph{\z} ist speziell für IBM Z~Systems entwickelt. Es ist darauf ausgelegt, besonders robust zu sein und zeichnet sich durch seine Anwendungen aus. Funktionen wie Scheduling oder Data-Allocation, die klassischerweise zu den Kernaufgaben eines \ac{os} gehören, werden auf dedizierte Anwendungen ausgelagert.


% \subsubsection{IWS}


% \subsubsection{IMS}


\subsubsection{Job}
Ein Job ist eine Aufgabe, die wiederum aus Teilaufgaben bestehen kann.  Jobs bilden die Grundlage für die Arbeit mit dem Mainframe. Sie bestehen aus einem oder mehreren Befehlen, die Programme ausführen oder Daten ent-\,/\,laden. Standardmäßig werden sie in \ac{jcl} geschrieben. \ac{jcl} ist eine Skriptsprache über die die Rahmenbedingungen einer Aufgabe, z.B. Environtment oder In-\,/\,Output, definiert werden können. JCL selbst ist nicht zum Programmieren geeignet, da sie über keine Programmierfunktionen wie Schleifen oder andere Kontrollstrukturen verfügt.
Um ausgeprägte Programme zu erstellen wird \ac{rex} verwendet. Hierbei handelt es sich um eine interpretierte Programmiersprache, die für Programme mit komplexen Abläufen geeignet ist. \ac{rex} kann auch innerhalb eines \ac{jcl}-Jobs verwendet werden.

%%%%%%%%%%%%%%%%%%%%%%%%%%%%%%%%%%%%%%%%%%%%%%%%%%
\section[Db2]{Datenbankmanagementsystem Db2 z/OS}
Db2 ist ein relationales \ac{dbms}, das speziell für den IBM Mainframe entwickelt ist. Es ist nativ gut verbunden mit anderen IBM Produkten wie 

\subsection{Datenorganisation}

\subsubsection[\ac{db}]{Datenbank}

\subsubsection[\ac{ts}]{Tablespace}

\subsubsection[\ac{tb}]{Table}


\subsection{Objekte}

\subsubsection{Indizes}

\subsubsection{LOB}
BLOB, CLOB
Trennung in Aux und Data-Table


\subsection{\ac{db}-Subsystem}
Instanz einer 