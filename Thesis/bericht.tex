\documentclass[
   ngerman          % neue deutsche Rechtschreibung
  ,a4paper          % Papiergrösse
% ,twoside          % Zweiseitiger Druck (rechts/links)
% ,10pt             % Schriftgrösse
% ,11pt
  ,12pt
  ,pdftex
%  ,disable         % Todo-Markierungen auschalten
]{report}

%  \usepackage[ansinew]{inputenc}   % Für Microsoft Windows
\usepackage[utf8]{inputenc}        % UTF-8 codierte Dateien
                                   % Dieses Dokument ist unter Unix erstellt, daher
                                   % wird diese Input-Codierung benutzt.
\usepackage{bericht}

\csname endofdump\endcsname

\newcommand{\Autor}{Patrick Frey}
\newcommand{\MatrikelNummer}{3946606}
\newcommand{\Kursbezeichnung}{tinf20b2}

\newcommand{\FirmenName}{Atruvia AG}
\newcommand{\FirmenStadt}{Karlsruhe}
\newcommand{\FirmenLogoDeckblatt}{\includegraphics[width=8cm]{Pictures/Atruvia-Logo.png}}

% Falls es kein Firmenlogo gibt:
%  \newcommand{\FirmenLogoDeckblatt}{}

\newcommand{\BetreuerFirma}{Rolf Merkle}
\newcommand{\BetreuerDHBW}{Michael Vetter}

% Wird auf dem Deckblatt und in der Erklärung benutzt:
%\newcommand{\Was}{Projekt-/Studien-/Bachleorarbeit}
%\newcommand{\Was}{Projektrarbeit}
%\newcommand{\Was}{Studienarbeit}
\newcommand{\Was}{Bachleorarbeit}

\newcommand{\Titel}{Recovery für Db2}
\newcommand{\AbgabeDatum}{4. September 2023}

\newcommand{\Dauer}{13 Wochen}

% \newcommand{\Abschluss}{Bachelor of Engineering}
\newcommand{\Abschluss}{Bachelor of Science}

%\newcommand{\Studiengang}{Informatik / Informationstechnik}
 \newcommand{\Studiengang}{Informatik / Angewandte Informatik}

\hypersetup{%%
  pdfauthor={\Autor},
  pdftitle={\Titel},
  pdfsubject={\Was}
}

% Wenn \includeonly{..} benutzt wird, werden nur diese Kaptitel ausgegeben.
\includeonly{
  abk
 ,kapitel1
 ,kapitel2
 ,changelog
}

% Benutzt man das "biblatex"-Paket, dann muß das hier stehen:
% siehe auch die mit BIBLATEX markierten Zeilen in bericht.sty
\bibliography{bericht}

\begin{document}

\begin{titlepage}
\begin{center}
\vspace*{-2cm}
\FirmenLogoDeckblatt\hfill\includegraphics[width=4cm]{dhbw-logo}\\[2cm]
{\Huge \Titel}\\[1cm]
{\Huge\scshape \Was}\\[1cm]
{\large für die Prüfung zum}\\[0.5cm]
{\Large \Abschluss}\\[0.5cm]
{\large des Studienganges \Studiengang}\\[0.5cm]
{\large an der}\\[0.5cm]
{\large Dualen Hochschule Baden-Württemberg Karlsruhe}\\[0.5cm]
{\large von}\\[0.5cm]
{\large\bfseries \Autor}\\[1cm]
{\large Abgabedatum \AbgabeDatum}
\vfill
\end{center}
\begin{tabular}{l@{\hspace{2cm}}l}
Bearbeitungszeitraum	         & \Dauer 			\\
Matrikelnummer	                 & \MatrikelNummer		\\
Kurs			         & \Kursbezeichnung		\\
Ausbildungsfirma	         & \FirmenName			\\
			         & \FirmenStadt			\\
Betreuer der Ausbildungsfirma	 & \BetreuerFirma		\\
Gutachter der Studienakademie	 & \BetreuerDHBW		\\
\end{tabular}
\end{titlepage}

%%%%%%%%%%%%%%%%%%%%%%%%%%%%%%%%%%%%%%%%%%%%%%%%%%%%%%%%%%%%%%%%%%%%%%%%%%%%%%%

\input{erklaerung.tex}

%%%%%%%%%%%%%%%%%%%%%%%%%%%%%%%%%%%%%%%%%%%%%%%%%%%%%%%%%%%%%%%%%%%%%%%%%%%%%%%

\begin{abstract}
Dieses \LaTeX-Dokument kann als Vorlage für einen Praxis- oder Projektbericht, eine Studien- oder
Bachelorarbeit dienen.

Zusammengestellt von Prof.\,Dr.\,Jürgen Vollmer \email{juergen.vollmer@dhbw-karlsruhe.de}\\
\url{https://www.karlsruhe.dhbw.de}. Die jeweils aktuellste Version dieses \LaTeX-Paketes ist immer
auf der \emph{FAQ-Seite} des Studiengangs Informatik zu finden:
\url{https://www.karlsruhe.dhbw.de/inf/studienverlauf-organisatorisches.html} $\to$ \emph{Formulare und Vorlagen}.

\centering Stand \verb+$Date: 2020/03/13 15:07:45 $+
\end{abstract}

\newpage
\tableofcontents           % Inhaltsverzeichnis hier ausgeben
\listoffigures             % Liste der Abbildungen
\listoftables              % Liste der Tabellen
\lstlistoflistings         % Liste der Listings
\listofequations           % Liste der Formeln

% Jetzt kommt der "eigentliche" Text
\include{abk}              % Abkürzungsverzeichnis
%%%%%%%%%%%%%%%%%%%%%%%%%%%%%%%%%%%%%%%%%%%%%%%%%%%%%%%%%%%%%%%%%%%%%%%%%%%%%%
%% Descr:       Vorlage für Berichte der DHBW-Karlsruhe, Ein Kapitel
%% Author:      Prof. Dr. Jürgen Vollmer, vollmer@dhbw-karlsruhe.de
%% $Id: kapitel1.tex,v 1.24 2020/03/13 16:02:34 vollmer Exp $
%% -*- coding: utf-8 -*-
%%%%%%%%%%%%%%%%%%%%%%%%%%%%%%%%%%%%%%%%%%%%%%%%%%%%%%%%%%%%%%%%%%%%%%%%%%%%%%%

\chapter{Einleitung}

\section{Dateien}
Diese Vorlage umfasst folgende Dateien:
\begin{description}
\item[bericht.tex] Die Haupt-\TeX-Datei. Hier werden die Einstellungen für das
     Deckblatt vorgenommen.
\item[bericht.sty] Die benötigten \LaTeX-Pakete werden hier aufgelistet. Eigene Macros definiert.
\item[bericht.bib] Die Bib\TeX\ "`Datenbank"' für die Literaturreferenzen.
\item[abk.tex] \LaTeX-Datei, welche Abkürzungen definiert.
\item[kapitel1.tex] \LaTeX-Datei für das 1. Kapitel.
\item[kapitel2.tex] \LaTeX-Datei für das 2. Kapitel.
\item[dhbw-logo.png] Das Logo der DHBW-Karlsruhe.
\item[lowe.png] Das \LaTeX-Maskottchen.
\item[Makefile] Zum Erzeugen der PDF-Ausgabe.
\item[Pakete] Das Verzeichnis enthält einige Pakete, die u.\,U.\,unter \emph{Unix} nicht installiert
     sind. Wenn \LaTeX\ also darüber beklagt, daß Pakete fehlen, folgen Sie den Installationsanweisungen
     der Pakete. Prüfen Sie, ob es neuere Versionen der Pakte gibt. In der Datei
     \texttt{bericht.sty} sind entsprechende Links auf die Quellen im Internet angegeben.

     Wenn Sie unter \emph{Microsoft Windows} bei der Installation
     \enquote{Install missing packages on the fly $\longrightarrow$ YES} ausgewählt haben,
     werden fehlende Pakete automatisch installiert.
\end{description}

%%%%%%%%%%%%%%%%%%%%%%%%%%%%%%%%%%%%%%%%%%%%%%%%%%%%%%%%%%%%%%%%%%%%%%%%%%%%%%%

\section{Erzeugen der PDF-Dateien}

\subsection{Unix + Kommandozeile}
Die Programmaufrufe zum Erzeugen der \Def{PDF-Datei} unter \emph{Unix}
sind im \texttt{Makefile} angegeben. Im Wsentlichen ruft man in der Konsole das Kommando
\texttt{pdflatex bericht}. Damit alle Referenzen innerhalb des Textes, die Seitennummern,
die Literaturreferenzen etc.\,korrekt ausgegeben werden, muss man \LaTeX mindestens dreimal hintereinander aufrufen.
\begin{verbatim}
   pdflatex bericht
   bibtex   bericht
   makeindex -s bericht.ist bericht
   pdflatex bericht
   pdflatex bericht
\end{verbatim}
Dieser vollständgige Zyklus ist aber für's \enquote{Probelesen} nicht nötig.
\texttt{bibtex}  erzeugt die Lieteraturreferenzen, \texttt{makeindex} erstellt den Index.

\subsection{Andere}
Unter \emph{Microsoft Windows} öffnen Sie die Datei \emph{bericht.tex} im \emph{TexnicCenter}.
In vielen Betriebsystemen gibt es auch graphische Oberflächen zur Erstellung von Texten mit \LaTeX,
diese erzeugen dann die PDF-Dateien -- ebenfalls durch Aufruf eines entsprechenden
Konsolenprogrammes, allerdings \enquote{unsichtbar} für den Benutzer.

\subsection{Geht's nicht etwas fixer? Eigene Formatdatei}
Das Einlesen aller eingebundenen Pakete pro Aufruf von \texttt{pdflatex} kann mitunter
\enquote{etwas dauern}. Dies lässt sich beschleunigen, indem man eine eigene \enquote{Formatdatei}
\index{Formatdatei} \texttt{bericht.fmt} erzeugt, diese enthält ein vorkomplierte \enquote{Version}
der Pakete. Damit \texttt{pdflatex} diese vorkompilierte Datei benutzt, muss in der ersten Zeile der
\texttt{bericht.tex} Datei folgende Zeile stehen:
\begin{verbatim}
%&bericht
\end{verbatim}
gefolgt von einer Leerzeile. Existiert die Datei \texttt{bericht.fmt} nicht, werden die Pakete
\enquote{wie üblich} einzeln eingebunden.

Damit \texttt{pdflatex} \enquote{weiss} was alles vorübersetzt werden soll, muss in
\texttt{bericht.tex} folgende Zeile stehen
\begin{verbatim}
\csname endofdump\endcsname
\end{verbatim}
ACHTUNG, wenn man eine eigene Formatdatei benutzt, werden Änderungen an \texttt{bericht.sty}
erst wirksam, wenn die Format-Datei neu erzeugt wurde!
Genauer alle Änderungen, die textuell vor  der Zeile \texttt{$\dots$ endofdump $\dots$} stehen,
werden erst wirksam, wenn die Formatdatei neu erzeugt wurde

Das Kommando zum Erzeugen der Formatdatei lautet:
\begin{verbatim}
   pdflatex -ini -jobname=bericht  "&pdflatex" mylatexformat.ltx bericht.tex
\end{verbatim}
Weitere Infos finden Sie auf den hier\footnote{
\url{https://tex.stackexchange.com/questions/79493/ultrafast-pdflatex-with-precompiling} und\\
\url{https://ctan.org/pkg/mylatexformat}}.

%%%%%%%%%%%%%%%%%%%%%%%%%%%%%%%%%%%%%%%%%%%%%%%%%%%%%%%%%%%%%%%%%%%%%%%%%%%%%%%

\section{Einfügen von Bildern und Querverweise im Text}

\index{Bilder}


Die Benutzung des \texttt{varioref}-Paketes macht das Benutzen von Referenzen einfacher.

%%%%%%%%%%%%%%%%%%%%%%%%%%%%%%%%%%%%%%%%%%%%%%%%%%%%%%%%%%%%%%%%%%%%%%%%%%%%%%%

\section{Literaturreferenzen}

\LaTeX\ \cite{lamport.1995a} basiert auf \TeX \cite{knuth.1984a}.
Die Literaturreferenzen werden von Bib\TeX verwaltet.

Hier ein Beispiel des Zitierens von Web-Seiten
\cite{dante.2010a} ist der Anlaufpunkt für \LaTeX\ in Deutschland.

URLs zitieren kann man so \cite{dante.2010a} machen.

\section{Literaturreferenzen mit dem Bib\LaTeX-Paket}

\index{Literaturreferenz}
Das Bib\LaTeX-Paket erlaubt eine deutlich komfortableren Zugriff auf Einträge der
BiB\TeX-"`Datenbank"' als die einfachen Bib\TeX-Stile. Allerdings ist das \texttt{bibtex}-Paket
nicht standard mässig installiert. Es muß zusammen mit dem \texttt{etoolbox}-Paket installiert
werden, s.\
\url{http://dante.ctan.org/tex-archive/help/Catalogue/entries/etoolbox.html} und\\
\url{http://dante.ctan.org/tex-archive/help/Catalogue/entries/biblatex.html}.


% Nur mit BIBLATEX
Ein Beispiel was man mit Bib\LaTeX\ machen kann (siehe auch \texttt{bericht.s}).

\citefullauthor{knuth.1984a} hat in seinem wegeweisenden Buch
\citetitle{knuth.1984a} aus dem Jahr \citeyear{knuth.1984a}
die Grundlagen von \TeX\ gelegt.

% nur mit BIBLATEX:
Nur die URL angeben: \citeurl{dante.2010a} oder URL mit Referenz:
\citeurlref{dante.2010a}, oder eben "`einfach"' wie oben gezeigt.

%%%%%%%%%%%%%%%%%%%%%%%%%%%%%%%%%%%%%%%%%%%%%%%%%%%%%%%%%%%%%%%%%%%%%%%%%%%%%%%

\section{Quellcodelistings}


%%%%%%%%%%%%%%%%%%%%%%%%%%%%%%%%%%%%%%%%%%%%%%%%%%%%%%%%%%%%%%%%%%%%%%%%%%%%%%%


Das Paket\todo{Was waren nochmal Pakete?} \texttt{todonotes} stellt das Makro\todo{Was sind \LaTeX\ Macros?}
\verb+\todo{...text....}+ zur Verfügung.

Das Macro \verb+\missingfigure{Da fehlt noch ein Bild}+ erzeugt
\missingfigure{Da fehlt noch ein Bild}.

\todo[color=red,inline]{Das Handbuch \texttt{todonotes} lesen!}

Am Ende des Dokuments wird die Liste aller ToDo's mit \verb+\listoftodos+ ausgegeben\\
(siehe \texttt{bericht.tex}).

\noindent
Das Paket kennt folgende Optionen:
\begin{description}
\item[\texttt{disable}] ToDo's nicht anzeigen
\end{description}

%%%%%%%%%%%%%%%%%%%%%%%%%%%%%%%%%%%%%%%%%%%%%%%%%%%%%%%%%%%%%%%%%%%%%%%%%%%%%%%

\section{Indices}

Mit dem Paket \verb+makeinx+ und dem Macro \verb+\index+ können  leicht Indices erstellt werden.
Das Macro \verb+\Def{..}+ kann für definitinen benutzt werden.
z.\,B.\, Mit demm optionalen Argument wie in  \verb+\Def[Definition]{Definitionen}+
(\Def[Definition]{Definitionen}) können verschiedene Schreibweisen im text und Index angegeben
werden.
Weitere interessante Möglichkeiten sind:
\begin{itemize}
\item \verb+\index{Punkt!Unterpunkt}+ \index{Punkt!Unterpunkt}
\item \verb+\index{Verweis|see{Punkt}}+ \index{Verweis|see{Punkt}}
\end{itemize}

%%%%%%%%%%%%%%%%%%%%%%%%%%%%%%%%%%%%%%%%%%%%%%%%%%%%%%%%%%%%%%%%%%%%%%%%%%%%%%%

\section{Sachen, die mir Anwender geschickt haben}

\subsection{Erstellen eines Formelverzeichnises}
\textsc{Andy Nöltner} \url{ANoeltner@lstelcom.com}

\begin{equation}
hx = x \cdot \tan \alpha
\eqlabel{eq-hx-angle}{Berechnung Höhenunterschied Tx zu Rx}
\end{equation}

%%%%%%%%%%%%%%%%%%%%%%%%%%%%%%%%%%%%%%%%%%%%%%%%%%%%%%%%%%%%%%%%%%%%%%%%%%%%%%%
\endinput
%%%%%%%%%%%%%%%%%%%%%%%%%%%%%%%%%%%%%%%%%%%%%%%%%%%%%%%%%%%%%%%%%%%%%%%%%%%%%%%

\include{kapitel2}

% Ab hier beginnt der Anhang
\appendix
\addcontentsline{toc}{chapter}{Anhang}

\addcontentsline{toc}{chapter}{Index}
\printindex

\addcontentsline{toc}{chapter}{Literaturverzeichnis}

% Haben Sie das "biblatex"-Paket nicht installiert, benutzen Sie folgendes:
% Ohne das "biblatex"-Paket (s. bericht.sty) produziert folgendes
% "deutsche" Zitate in Literaturverzeichnissen gemaß der Norm DIN 1505,
% Teil 2 vom Jan. 1984.
% Die Zitatmarken werden alphabetisch nach Verfassern
% sortiert und sind durch abgekürzte Verfasserbuchstaben plus
% Erscheinungsjahr in eckigen Klammern gekennzeichnet.

% \bibliographystyle{alphadin}
% \bibliography{bericht}

%%%%%%%%%%%%%%%%%%%%%%%%%%%%%%%%%%%%%%%5
% BIBLATEX
% Benutzt man das "biblatex"-Paket, muß man folgendes schreiben:
\def\refname{Literaturverzeichnis}
\printbibliography
%%%%%%%%%%%%%%%%%%%%%%%%%%%%%%%%%%%%%%%5


\newpage
\addcontentsline{toc}{chapter}{Liste der ToDo's}
\listoftodos[Liste der ToDo's]


\end{document}
